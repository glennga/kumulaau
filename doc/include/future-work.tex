\chapter{Future Work}\label{ch:futureWork}
In this chapter, we discuss the next steps associated with our research.

\section{Evolutionary Simulator}\label{sec:evolutionarySimulator}
The focus of this project was to find an appropriate evolutionary simulator and model for a single population.
We believe that we have constructed a fast evolutionary simulator, but it could be more efficient.
As of now, the running time of our evolutionary simulator is dependent on $n$ and $N$.
$n$ must be large enough to account for the variance in this problem, but we can remove our reliance on $N$ by
drawing our mutations from some distribution instead of iterating from generation to generation.
This increases the noise associated with our mutation model, but may make our simulator much faster.

\section{Implementation Updates}\label{sec:implementationUpdates}
All code for this project is current written in Python 3.7.
We were able to get reasonable speedup using Numba to compile certain sections of our code (as opposed to only
interpreting it), but it might be of interest to move to a compiled language like C++ altogether.
It might also be of interest to start moving certain sections of the sampler (simulator, $\delta$, etc\ldots) to the
GPU.
This problem is highly parallelizable, but we are only running at most 20 tasks in parallel on the CPU.
The GPU would allow us to run thousands of tasks in parallel, (ideally) resulting in less runtime.

\section{Mutation Model}\label{sec:mutationModel}
We were able to obtain estimates for our mutation model, but we need to further verify that these parameters are
correct.
There exists three main areas of future work:
\begin{enumerate}
    \item Verifying the model using more populations.
        We were only able to use the Columbian populace here, but using the same methodology with other groups would
        give us different $\hat{c}$ and $\hat{d}$ values to compare to.
    \item Comparing the total mutation rate to that of other \texttt{GATA} microsatellites in different literature.
    \item Running our MCMC longer.
        As seen in the previous section, our trace was not the ideal shape but it showed hints toward convergence.
        The easiest solution here to increase our number of iterations $T_2$, which would show a straighter and thicker
        line if we have truly converged.
    \item Retrieving more statistics about our Metropolis sampler.
        We visually assessed Markov chain convergence by looking at the trace plot, but there exist other metrics
        that could be explored (see~\cite{cowlesMarkovChainMonte1996}).
        This would allow us to tune our sampler with more precision, and possibly find better estimates for $\hat{c}$
        and $\hat{d}$.
\end{enumerate}

\section{Demographic Models}\label{sec:demographicModels}
Once we are confident in our mutation model, we can extend our work toward using the same methodology with
\emph{multiple} populations.
The mutation model parameters then become hyperparameters to these demographic models as we try to estimate
additional parameters such as population size, admixture, and divergence times.
Below, we present several multi-population models from Stoneking and Krause that can be used to make various
inferences~\cite{stonekingLearningHumanPopulation2011}.
\begin{enumerate}
    \item Three population models: (Ancestors, African vs.\ Non African), (Ancestors, Modern Humans vs.\ Neanderthals),
        (Ancestors, Modern Humans vs.\ Denisovans).
        We can determine the population sizes of all populations and the time of each split.
    \item A five population model: Ancestors, African One, African Two, Non African One, Non African Two.
        We can determine the admixture between the African and Non African populations, as well as the time associated
        with each split.
    \item A seven population model: Ancestors, Modern Humans One, Modern Humans Two, Neanderthals One, Neanderthals Two,
        Denisovan One, Denisovan Two.
        Again, we can determine the admixture between each population as well as the time associated with each split.
\end{enumerate}
%
%\section{Summary}\label{sec:summary}
%Human inferences can be made by tracking variations in DNA.
%In this paper we went over how we can use microsatellite DNA, a form of genetic variation in humans, to infer
%human evolution.
%Our first approach was to use a forward simulator model to simulate generation to generation transitions, but we
%ended up using a backward simulator model because it required less individuals to simulate.
%Our two parameter $c, d$ mutation model was selected for its simplicity and flexibility.
%To estimate the most likely parameter values $\hat{c},\hat{d}$, we used an ABC-MCMC approach.
%Future work includes verifying these parameters further and extending our methodology to more demographic models.
