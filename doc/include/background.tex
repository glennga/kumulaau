\section{Background}\label{sec:b}
\subsection{Microsatellites vs. Single Nucleotide Polymorphisms}\label{subsec:mvsnp}
\subsubsection{Microsatellites}
\subsubsection{Single Nucleotide Polymorphisms}

\subsection{Simulating Evolution}\label{subsec:se}
In this section we cover how we represent microsatellites in this paper, the major different variations for simulation,
and the approach we take.

\subsubsection{Microsatellite Representation}
Let $i$ represent an integer associated with number of repeat units of a single microsatellite.
For example if we define our repeat unit as \emph{GATA}, the microsatellite below would represented as $i=4$.
\begin{equation*}
    \ldots GATAGATAGATAGATA \ldots
\end{equation*}
% TODO: Rose and Falush 1998
We constrain $i$ to exist between some lower bound on repeat units $\kappa$ and some upper bound $\Omega$.
Microsatellites are rarely longer than a few tens of repeat units $\Omega$, and a lower bound ensures that mutations can
actually occur.

% TODO: Wut
Let $\ell$ represent a set, or \emph{population} of microsatellites $\{ i_1, i_2, i_3, \ldots i_{2N} \}$ in a
human population of size $N$.
There exist two microsatellites per individual in a human population, but we relax the constraint..

\subsubsection{Forward vs. Backward Simulation}
There exists two main approaches toward which direction we should simulate toward: from the past to the present
(forward) or from the present to the past (backward).
In both cases, the goal is to create some evolutionary tree of $2N$ microsatellites that trace back to some common
ancestor.

In order to simplify this problem, we assume the Wright-Fisher population model:
\begin{enumerate}
    \item Our population of microsatellites is of constant size $2N$ for each generation.
    \item All individuals from some generation $t$ originate from the previous generation $t - 1$.
    \item There exists no selection bias, each microsatellite from the previous generation is equally likely to be
        chosen to exist in the next.
\end{enumerate}

\paragraph{Forward Simulation}
Let $\ell_t$ represent a microsatellite population at generation $t$, of size $|\ell_t| = 2N$.
For a population based simulation (see~\autoref{subsubsec:efbvcb}), we define the next generation $\ell_{t+1}$ by
randomly sampling $2N$ microsatellites from $\ell_t$.
\begin{equation}
    \ell_{t + 1} = \Call{MoveForward}{\ell_t}
\end{equation}
where \Call{MoveForward}{} is defined in~\autoref{alg:mf}.
\begin{algorithm}[t]
    \caption{Forward, population based simulation for a single generation.}\label{alg:mf}
    \begin{algorithmic}[1]
        \Function{MoveForward}{$\ell_t$}
            \State $\ell_{t + 1} \gets \emptyset$
            \For{$j \textbf{ to } 2N$}
                \State $\ell_{t + 1} \gets \ell_{t + 1} \cup \{ \Call{Mutate}{\text{single, random microsatellite from }
                    \ell_t }\}$
            \EndFor
            \State \textbf{return } $\ell_{t + 1}$
        \EndFunction
    \end{algorithmic}
\end{algorithm}

Starting from $\ell_0$, we evolve forward until all individuals in the current generation coalesce.
On average, this is $4N$ generations:
\begin{enumerate}
    \item In a Wright-Fisher population of $2N$ microsatellites, the probability that two microsatellites
        $i_{child1}$ and $i_{child2}$ in generation $t$ share the same parent $i_{parent}$ is $\frac{1}{2N}$.
    \item
\end{enumerate}


\paragraph{Backward Simulation}
Let $\ell_t$ represent a microsatellite population at generation $t$, of size $|\ell_t | = 2N$.
For a population based simulation (see~\autoref{subsubsec:efbvcb}), we define the previous generation $\ell_{t - 1}$ by
randomly sampling

\subsubsection{Effective Population Based vs. Coalescent Based}\label{subsubsec:efbvcb}
\paragraph{Effective Population Based}
\paragraph{Coalescent Based}

\subsection{Modeling Mutation}\label{subsec:mm}
In this section,

\subsubsection{Change in Repeat Units}
\subsubsection{Rate of Mutations}
\subsubsection{Contraction vs. Expansion}

\subsection{Algorithm for Microsatellite Evolution}\label{subsec:afme}
In this section, we describe
\subsubsection{Single Population Evolution}
\subsubsection{Multiple Population Evolution}