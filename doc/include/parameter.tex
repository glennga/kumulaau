\chapter{Parameter Estimation}\label{ch:parameterEstimation}
In this chapter, we discuss our approach toward parameter estimation.
Specifically, we aim to address the problem of maximum likelihood.

\section{Intractable Likelihood Functions}\label{sec:intractableLikelihoodFunctions}
\begin{figure}[t]
    \centering{\pgfplotsset{compat=1.5}
\pgfmathdeclarefunction{gauss}{2}{%
  \pgfmathparse{1/(#2*sqrt(2*pi))*exp(-((x-#1)^2)/(2*#2^2))}%
}

\begin{tikzpicture}
    \begin{axis}[
    width=1.15\linewidth, height=6.4cm,
    ylabel={Likelihood}, ymax=0.3,
    xlabel={Parameter Value}, xmin=0, xmax=10,
    samples=100, no markers, enlargelimits=false, legend style={at={(0.5,-0.35)},anchor=north},
    domain=0:25, ticks=none
    ]
    \addplot [very thick,blue!50!black] {gauss(5,1.5)};

    \addplot+[very thick, black, dashed] coordinates {(0, 0.26596152) (25, 0.26596152)};
    \end{axis}
\end{tikzpicture}}
    \caption{General figure depicting a smooth likelihood distribution (blue).
    The maximum likelihood point of this distribution is the intersection of the black dashed line and the blue line.
    }\label{fig:likelihoodGeneral}
\end{figure}

The \emph{likelihood} of some parameter set $\theta$ is the joint probability that $\theta$ produces some observed data
$\mathcal{D}$.
Let $\mathcal{M} : \theta, V \mapsto \mathcal{D}'$ define a function that maps a parameter set $\theta$ and some
random variables $V$ to generated data $\mathcal{D}'$.
Given that $V$ is randomly distributed, $\mathcal{M}(\theta, V)$ may produce different $\mathcal{D}'$ for the same
parameter set $\theta$.
In other words, $\mathcal{M}$ represents some stochastic simulator model which produces some random data.
Given observations $\mathcal{D}$, the likelihood of parameters $\theta$ is the probability that our generated data
equals our observed data~\cite{lintusaariFundamentalsRecentDevelopments2017}.
\begin{equation}\label{eq:likelihood1}
    \mathcal{L}(\theta) = \Pr(\mathcal{D} = \mathcal{M}(\theta, V))
\end{equation}

If $\mathcal{L}$ is continuous, one technique to calculate a maximum likelihood for our parameters $\hat{\theta}$
involves finding the critical points of $\mathcal{L}$ (where $\frac{d\mathcal{L}}{d\theta} = 0$) and choosing the point
$\hat{\theta}$ with the largest value of $\mathcal{L}$.
In~\autoref{fig:likelihoodGeneral}, the maximum of such a likelihood is displayed at the intersection of the blue and
black dashed lines.
There are however, several problems with this approach:
\begin{enumerate}
    \item This assumes that we can explicitly express $\mathcal{L}$.
        For simulator based $\mathcal{M}$ like ours, this is not trivial to do.
    \item $\theta$ may represent multiple parameters to differentiate over.
        Even if we could express our likelihood, calculating this multi-dimensional derivative analytically may be
        not feasible.
    \item With high dimensional data $\mathcal{D}$ and $\mathcal{D}'$, an exact match
        $\mathcal{D} = \mathcal{D}'$ may not occur within a reasonable amount of time.
\end{enumerate}
Consequently, we must look into other approaches to \textit{infer} $\mathcal{L}$.
The two problems this section aims to address are (a) how to efficiently compute $\mathcal{L}$ for a single $\theta$
and (b) how to infer $\mathcal{L}$ for all $\theta$.

\section{Approximate Bayesian Computation}\label{sec:approximateBayesianComputation}
In this section, we discuss an approximate method to compute the likelihood $\mathcal{L}$ for some parameter set
$\theta$.

As previously mentioned, to find $\mathcal{L}$ is to find the joint probability that our model and parameters produce
the observed data.
We expand~\autoref{eq:likelihood1} to make this explicit for $n$ sets of observed data.
$\theta$ remains the same, however the randomness in $V$ may result in different probabilites.
\begin{equation}
    \mathcal{L}(\theta) = \Pr\left(\mathcal{M}(\theta, V) = \mathcal{D}_1 \right)
    \Pr\left(\mathcal{M}(\theta, V) = \mathcal{D}_2 \right) \ldots
    \Pr\left(\mathcal{M}(\theta, V) = \mathcal{D}_n \right)
\end{equation}

Let $\mathcal{D}$ represent the set of observed samples from the Columbian populace.
How do we determine the probability of our model generating some $\mathcal{D}_i \in \mathcal{D}$?
Here, we take a frequentist approach and perform the following steps:
\begin{enumerate}
    \item Run our simulator once to get $\mathcal{D}'_1$.
    \item Check if $\mathcal{D}'_1$ matches $\mathcal{D}_1$.
    \item Repeat steps 1 and 2 some arbitrary number of times $T_1$, for different simulated samples but the same
        observed sample.
        We define $\Pr\left( \mathcal{M}(\theta, V) = \mathcal{D}_1 \right)$ as the frequency of exact matches.
    \item Repeat step 3 for all observed samples to get $\Pr\left( \mathcal{M}(\theta, V) = \mathcal{D}_2 \right)$,
        \ldots, $\Pr\left( \mathcal{M}(\theta, V) = \mathcal{D}_n \right)$.
\end{enumerate}
To find the $\mathcal{L}$ is to multiply all probabilities together.

This approach seems simple enough, but there exists one caveat: the frequency of exact matches is too low to interpret
anything meaningful.
The microsatellite repeat length set $\mathbb{M}$ consists of roughly 30 elements, meaning that roughly 30 dimensions
must exactly match.
In addition to this, each frequency resides as a real number in $[0, 1]$.
The solution here is a technique known as \emph{Approximate Bayesian Computation}, or ABC for short.
ABC has two parts: the use of \emph{approximate} matches and the use of summary
statistics~\cite{lintusaariFundamentalsRecentDevelopments2017}.
In the following sections, I discuss how we define approximate, what summary statistics are, and why we are not using
them for this problem.

\begin{figure}[t]
    \centering{\pgfplotsset{compat=1.5}
\pgfmathdeclarefunction{gauss}{2}{%
  \pgfmathparse{1/(#2*sqrt(2*pi))*exp(-((x-#1)^2)/(2*#2^2))}%
}
\pgfmathdeclarefunction{smallergauss}{2}{%
  \pgfmathparse{0.03/(#2*sqrt(2*pi))*exp(-((x-#1)^2)/(2*#2^2))}%
}

\begin{tikzpicture}
    \begin{axis}[
    width=\linewidth, height=6.4cm,
    ylabel={Likelihood}, ymax=0.03,
    xlabel={Parameter Value}, xmin=0, xmax=40,
    samples=100, no markers, enlargelimits=false, legend style={at={(0.5,-0.35)},anchor=north},
    domain=0:40, ticks=none
    ]
    \addplot [very thick,blue!50!black] {smallergauss(20,1.5)};
    \addplot [very thick,red!50!black] {gauss(20,20)};

    \end{axis}
\end{tikzpicture}}
    \caption{General figure depicting a true likelihood surface (blue) and an approximate surface (red).
    The true likelihood may not be found in a reasonable amount, where as the approximate likelihood is wider but
    taller (making it more tractable).
    }\label{fig:approximateLikelihood}
\end{figure}

\subsection{Approximate Matches: $\epsilon$}\label{subsec:approximateMatches}
Let us dissect $\mathcal{D}$ and $\mathcal{D}'$ further.
A given $\mathcal{D}_i \in \mathcal{D}$ and $\mathcal{D}'_i \in \mathcal{D}'$ define $| \mathbb{M} |$-sized tuples,
whose values exist in $[0, 1]$.
To compare a given $\mathcal{D}_i$ and $\mathcal{D}'_i$ is to iterate through each element in both tuples and verify
their equality.
Let $\delta : \mathcal{D}_i,\mathcal{D}'_i \mapsto [0, 1]$ represent some function that accepts the observed and
generated sample, and outputs some distance between 0 and 1.
The comparison between our observed and generated samples is given by $\delta(\mathcal{D}_i, \mathcal{D}'_i)$.

We were only able to explore one $\delta$ function: the angular (or Cosine) distance.
The angular distance treats $\mathcal{D}_i$ and $\mathcal{D}'_i$ as $| \mathbb{M} |$-dimensional vectors
and aims to quantify some difference between the two.
An output of 0 indicates that both samples are completely similar, while an output of 1 indicates that two vectors
are maximally dissimilar (orthogonal).
The angular distance $\delta_A$ is defined as such (normalized to make $\delta_A$ a proper distance
metric)~\cite{chaComprehensiveSurveyDistance2007a} for two samples $\mathcal{D}_i, \mathcal{D}'_i$:
\begin{equation}
    \delta_A(\mathcal{D}_i, \mathcal{D}'_i) = \frac{2}{\pi} \arccos \left(
    \frac{\sum_{\ell=\kappa}^{\Omega} \mathcal{D}_i[\ell] \cdot \mathcal{D}'_i[\ell]}{
        \sqrt{\sum_{\ell=\kappa}^{\Omega} \left(\mathcal{D}_i[\ell]\right)^2} \cdot
        \sqrt{\sum_{\ell=\kappa}^{\Omega} \left(\mathcal{D}'_i[\ell]\right)^2}
    } \right)
\end{equation}
where $\mathcal{D}_i[\ell]$ and $\mathcal{D}'_i[\ell]$ represent the frequency of repeat length $\ell$ for the observed
and generated samples respectively.

With some distance is quantified, the next step is defining what ``approximate'' means.
According to ABC, two samples $\mathcal{D}_i$ and $\mathcal{D}'_i$ are approximate matches if their distance falls below
some threshold $\epsilon \in [0, 1]$~\cite{marjoramMarkovChainMonte2003}:
\begin{equation}
    \left(\delta(\mathcal{D}_i, \mathcal{D}'_i) < \epsilon \right) \Leftrightarrow
    \left(\mathcal{D}_i \text { and } \mathcal{D}'_i
    \text{ are approximate matches } \right)
\end{equation}
By increasing $\epsilon$, the frequency of exact observed -- generated matches increases as well.
By making this problem more tractable though, we increase the noise associated with drawing from $\mathcal{L}$.
This results in a flatter curve, as seen in~\autoref{fig:approximateLikelihood}.

The next question that follows is ``How do we know which $\epsilon$ to use?''
If $\epsilon$ is too small, the problem becomes intractable.
If $\epsilon$ is too large, we draw values that are not representative of the original distribution.
There is no clear answer to this question, and Lintusaari et.\ al.\ states this choice is typically made by
experimenting with different $(\mathcal{D}, \theta)$ pairs~\cite{lintusaariFundamentalsRecentDevelopments2017}.
We define our threshold $\epsilon$ as a \emph{hyperparameter}, a parameter we must specify to find the parameters of
interest $\theta$.

\subsection{Dimension Reduction: Summary Statistics}\label{subsec:dimensionReductionSummaryStatistics}
An alternative to using a distance function $\delta$ that deals with $| \mathbb{M} |$-dimensional points is to use a
function that reduces, or \emph{summarizes} the data into two points $\bar{\mathcal{D}}, \bar{\mathcal{D}'}$ of lower
dimensionality and finds a distance between both summarized points.
We specify a distance function $\delta_S$ that performs a transformation using functions
$h : \mathcal{D}_i \mapsto \bar{\mathcal{D}_i}$ and $h' : \mathcal{D}'_i \mapsto \bar{\mathcal{D}'_i}$ as such:
\begin{equation}
    \bar{\delta_S}(\mathcal{D}_i, \mathcal{D}'_i)  = \bar{\delta}(h(\mathcal{D}_i), h'(\mathcal{D}'_i))
\end{equation}
A common choice for $h, h'$ is the mean or median.
For us, it may make sense to use the focal length computation $\hat{\ell}$ as our summary statistic.

Using summary statistics avoids the curse of dimensionality (see~\cite{bellmanDynamicProgramming2013}) for high
dimension distance functions, but this adds yet another item we must specify: ``Which summary statistic is the best?''.
If we summarize our data wrong, we again run into the problem of drawing values that do not represent our original
distribution.
We ran several trials without reducing our dimensionality and have not run into any problems thus far.
To reduce noise, we are only using the angular distance $\delta_A$ for $| \mathbb{M} |$-dimensional vectors.

\section{Markov Chain Monte Carlo}\label{sec:markovChainMonteCarlo}
In this section, we discuss the Markov Chain Monte Carlo (MCMC) approach to approximating a likelihood function.

\begin{figure}[t]
    \centering{\pgfplotsset{compat=1.5}
\pgfmathdeclarefunction{gauss}{2}{%
  \pgfmathparse{1/(#2*sqrt(2*pi))*exp(-((x-#1)^2)/(2*#2^2))}%
}

\begin{tikzpicture}
    \begin{axis}[
    width=1.15\linewidth, height=6.4cm,
    ylabel={Likelihood}, ymax=0.3,
    xlabel={Parameter}, xmin=0, xmax=10,
    samples=100, no markers, enlargelimits=false, legend style={at={(0.5,-0.35)},anchor=north},
    domain=0:25, ticks=none
    ]
    \addplot [very thick,blue!50!black] {gauss(5,1.5)};
    \addplot+[very thick, black, dashed] coordinates {(2.5, 0.06631809) (2.5, 0.16131382) (3.5, 0.16131382)};
    \addplot+[very thick, black, dashed] coordinates {(3.5, 0.16131382) (3.5, 0.25158882) (4.5, 0.25158882)};
    \addplot+[very thick, red, dashed] coordinates {(4.5, 0.25158882) (4.0, 0.25158882) (4.0, 0.21296534)};

    \end{axis}
\end{tikzpicture}}
    \caption{General figure depicting the random walk of Metropolis sampler around some likelihood surface.
    }\label{fig:metropolisAlgorithm}
\end{figure}

\subsection{Monte Carlo}\label{subsec:monteCarlo}
In~\autoref{sec:approximateBayesianComputation}, we learned how to determine the likelihood of a single point $\theta$.
We are now interested in most likely $\theta$, out of all possible parameter values $\Theta$.
If $\Theta$ was discrete and countable, one approach we could take is to perform an exhaustive search and select
$\hat{\theta}$ such that it holds the highest possible likelihood:
\begin{equation}
    \mathcal{L}(\hat{\theta}) = \text{max}\left( \{ \mathcal{L}(\theta \mid \theta \in \Theta \} \right)
\end{equation}
Our $\Theta$ however is neither discrete nor countable.
As mentioned in~\autoref{sec:intractableLikelihoodFunctions}, we are also unable to solve this problem analytically
because we cannot explicitly express $\mathcal{L}(\theta)$.
The solution Monte Carlo algorithms propose is choosing $\theta$ randomly.

Knowing that we must choose $\theta$ randomly, we must now decide how to draw $\theta$.
Let $p : \Theta \rightarrow [0, 1]$ define a probability distribution that determines how we draw $\theta$.
$p$ is more commonly known as a \emph{prior distribution}, and allows us to insert any prior beliefs we have about our
likelihood before finding $\mathcal{L}(\theta)$ itself.
In Bayesian inference, the characterization of the uncertainty of some $\theta$ given observations $\mathcal{D}$ is
given by another distribution known as the
\emph{posterior distribution}~\cite{lintusaariFundamentalsRecentDevelopments2017}:
\begin{equation}
    \Pr(\theta \mid \mathcal{D}) \propto \mathcal{L}(\theta) p(\theta)
\end{equation}
The posterior is proportional to likelihood function we are trying to find.
If we can explicitly express our posterior, then the $\hat{\theta}$ that maximizes $\Pr(\theta \mid \mathcal{D})$ also
maximizes $\mathcal{L}(\theta)$.

\subsection{Markov Chain Monte Carlo}\label{subsec:markovChainMonteCarlo}
With the Monte Carlo strategy, our general procedure now becomes:
\begin{enumerate}
    \item Draw $\theta_i$ from our prior $p(\theta)$.
    \item Determine the likelihood of this point $\mathcal{L}(\theta_i)$.
    \item Repeat until we have a representative set of samples.
    \item Fit our samples to a curve, and return the $\hat{\theta}$ that maximizes $\mathcal{L}(\theta)$.
\end{enumerate}
The one problem we encounter here though, is our reliance on the prior.
A misinformed prior will produce samples that are not representative of the posterior.
As an example if $E(p(\theta)) = 500$ but our posterior is centered around $\theta = 1$, we will end up with a small or
nonexistent posterior.
\emph{Markov Chain Monte Carlo} methods solve this by sampling more often from regions of higher likelihood.

Markov Chain Monte Carlo methods work by constructing a Markov chain such that the posterior distribution is its
equilibrium distribution.
For simplicity, we assume $\Theta$ is discrete in the following explanations.
We start by defining a $\Theta$-sized vector $Y$, indexed by all distinct $\theta \in \Theta$ and whose values represent
probabilities associated with each $\theta$.
$Y$ is said to be a distribution here.
Next we define a matrix $G$ of size $\Theta \times \Theta$, whose entries describe the probability of transitioning
from one $\theta$ (row) to another $\theta$ (column).
$G$ is known as a \emph{transition matrix}.
We are able to move from distribution $Y^{(i)}$ to $Y^{(i + 1)}$ using this transition matrix:
\begin{equation}
    Y^{(i + 1)} = Y^{(i)} G
\end{equation}
$Y$ is said to be at \emph{equilibrium} if applying our transition matrix does not change the distribution:
\begin{equation}
    Y^{(i)} = Y^{(i)} G
\end{equation}
The goal of MCMC is to draw samples $X$ from $Y$ such that $Y$ represents $\mathcal{L}(\theta)$.
Let $X$ represent this \emph{chain} of states, or parameters $\theta$, that satisfy the following
conditions to draw from such a distribution~\cite{hanadaMarkovChainMonte2018}:
\begin{enumerate}
    \item \emph{$X$ is a Markov chain}.
        The probability of obtaining $X_i \in X$ from $X_{i - 1} \in X$ does not depend on any other configuration
        $X_{i - 2}, X_{i - 3}, \ldots, X_0$ other than $X_{i - 1}$ itself.
    \item \emph{$X$ is irreducible}.
        This states that we are able to travel to all of $\Theta$ from any given $\theta$ in a finite number of
        transitions.
    \item \emph{$X$ is aperiodic for all configurations}.
        A state $\theta$ is aperiodic if there exists a $i$ such that for all $j \geq i$:
        \begin{equation}
            \Pr \left( X_{j} = \theta \mid X_0 = \theta \right) > 0
        \end{equation}
    \item \emph{$X$ is positive recurrent}.
        The states that the expected number of transitions to move to back to the same state is finite.
\end{enumerate}
MCMC comprises a class of algorithms that are able to produce $X$ satifying these conditions.

\subsection{Metropolis Algorithm}\label{subsec:metropolisAlgorithm}
As per our last section, we want to generate some chain of states $X$ such that the given conditions are satisfied.
In this section, we describe the Metropolis algorithm-- a procedure that is able to generate a chain of states such that
these conditions are met.
There are three main steps to the Metropolis algorithm:
\begin{enumerate}
    \item \emph{Proposal}: We define some function $g : \Theta \rightarrow \Theta$ which generates a new $\theta_i$
        given an old $\theta'$.
        This relates to the transition matrix $G$ from~\autoref{subsec:markovChainMonteCarlo}, and constructs different
        $G$ for different values of $\theta_i$.
        The Metropolis algorithm is a special case of the \emph{Metropolis-Hastings algorithm} in which this proposal is
        symmetric.
    \item \emph{Calculate}: We determine the acceptance ratio $\alpha$, which is a ratio of the proposed $\theta_i$ to
        to the old $\theta'$.
    \begin{equation}
        \alpha = \frac{\mathcal{L}(\theta_i)}{\mathcal{L}(\theta)}
    \end{equation}
    \item \emph{Accept}: We save $\theta_i$ and $\mathcal{L}(\theta_i)$ to our collection of states if and only if
        $\theta_i$ is more likely than the old $\theta'$ \emph{or} the ratio of proposed to old likelihoods is greater
        than some uniform random variable $U(0, 1)$.
        If this is not true, then we go back to step (1) until we have run $T_2$ iterations.
\end{enumerate}

\begin{algorithm}[t]
    \SetAlgoLined
    \DontPrintSemicolon
    \Fn{MetropolisSampler \ {$(T_2, \mathcal{L}, \theta_1, g)$}} {
        \KwIn{number of algorithm iterations $T_2$, likelihood function $\mathcal{L}$, initial state $\theta_1$,
        proposal function $g$}
        \KwOut{samples from our posterior distribution $\Pr(\theta \mid \mathcal{D})$}
        $X \gets \emptyset$, $\theta' \gets \theta_1$ \;
        \For{$i \gets 1$ \KwTo $T_2$}{
            $\theta_i \gets g(\theta')$ \;

            \If{$\mathcal{L}(\theta_i) \cdot \mathcal{L}(\theta')^{-1} \leq \ \sim U(0, 1)$}{
                $\theta' \gets \theta_i$ \;
                $X \gets X \cup \{ \theta_i \}$ \;
            }
        }

        \Return $X$ \;
    }
    \textbf{end} \;
    \caption{The Metropolis algorithm, used to produce samples from a posterior distribution proportional to our
    likelihood.}
    \label{alg:metropolis}
\end{algorithm}

\section{Maximum Likelihood Estimation}\label{sec:maximumLikelihoodEstimation}
In this section, we describe a high level view of our approach to maximum likelihood estimation.
Using the Metropolis sampler, we are able to get a collection of states from a distribution proportional to our
likelihood $X$.
We now want to tie this back into our original question: ``Which parameter $\theta \in \Theta$ is the most likely to
produce our observations $\mathcal{D}$? ''
Given states $X$ from our posterior, how can we determine the most likely point $\theta$?

For this project, each $\theta$ represents a 2-tuple of $c$ and $d$.
We first construct two sets $C$ and $D$, which consist of each $c$ element and the corresponding $\mathcal{L}(\theta)$
for all $x \in X$:
\begin{equation}
    \begin{aligned}
        C &= \left(\mathcal{L}(\theta\right), c) \mid c \in \theta \land \theta \in X \\
        D &= \left(\mathcal{L}(\theta\right), d) \mid d \in \theta \land \theta \in X
    \end{aligned}
\end{equation}
If we assume our posterior for $c$ is normally distributed, we can assume the following function:
\begin{equation}
    \Pr(\theta \mid \mathcal{D}) = \frac{1}{\sigma\sqrt{2\pi}}\exp\left( -\frac{(c - \hat{c})^2}{2\sigma^2} \right)
\end{equation}
From here, a technique known as \emph{least squares regression} is used to find $\hat{c}, \sigma$.
This works by finding a $\hat{c}, \sigma$ such that the resulting curve minimizes the square of the errors to each
$c \in C$.
The mean of this normal distribution $\hat{c}$ represents the most likely point, given that our posterior is
proportional to our likelihood.
This is repeated for $d$ to get our most likely $\hat{d}$.
