\chapter{Data for Parameter Estimation}\label{ch:dataForParameterEstimation}
In this chapter, we discuss the observed data we are working with.
In the past several sections, we have worked on artificially generating a population of repeat lengths given several
parameters.
The data that we collect through observation must be in a similar format to this population of repeat lengths.
This allows us to determine the likelihood of some parameter set.

The data we are working with comes from ALFRED, the ALlele FREquency Database of humans from Yale.
The steps to extract the data from the ALFRED website are as follows:
\begin{enumerate}
    \item Navigate to https://alfred.med.yale.edu/alfred/index.asp.
    \item Type ``GATA'' into the search box to search for GATA repeats.
        We are using tetranucleotide repeats instead of other microsatellites because tetranucleotide repeats mutate
        more often than dinucleotide repeats.
    \item For each result, we saved the data if it was packaged as frequency against repeat length.
        Given a sample size, this data can easily be transformed into a population of repeat lengths.
        The inverse is also true (and is closer to what was performed): given a population of repeat lengths,
        a collection of frequency against repeat length is easy to obtain as well.
\end{enumerate}
%This gave us $\sim 14$ year old observations from the following populations:
%{\scriptsize
%    \begin{multicols}{4}
%        \begin{enumerate}
%            \item Ami
%            \item Atayal
%            \item Batak
%            \item Cambodians, Khmer
%            \item Filipino
%            \item Han
%            \item Japanese
%            \item Javanese
%            \item Kachari
%            \item Malaysians
%            \item Orang Asli
%            \item Papuan New Guinean
%            \item Samoans
%            \item So
%            \item Thai
%            \item Yadava
%            \item Argentine
%            \item Andalusian
%            \item Austrian
%            \item Catalans
%            \item Croatian
%            \item Germans
%            \item Maghrib
%            \item Mozambican
%            \item Portuguese
%            \item Spaniards
%            \item Galician
%            \item Afghan
%            \item African Americans
%            \item Agharia
%            \item Albanian
%            \item Apache
%            \item Arabs
%            \item Arrernte
%            \item Asian Australian
%            \item Athabaskan
%            \item Australian Aborigines
%            \item Australian, Caucasian
%            \item Azorian
%            \item Bahamian
%            \item Balinese
%            \item Bangladeshi
%            \item Basque
%            \item Bauri
%            \item Belarusian
%            \item Belgian
%            \item Berber
%            \item Bhumihar
%            \item Bosnian
%            \item Brahmin
%            \item Brazilian
%            \item Chamar
%            \item Chenchu
%            \item Chilean
%            \item Colombian
%            \item Costa Rican
%            \item Czech
%            \item Dhangars
%            \item Ecuadorian
%            \item Egyptians
%            \item Emirati
%            \item Equatorial Guinean
%            \item European Americans
%            \item Fang
%            \item Gabonese
%            \item Gond
%            \item Gope
%            \item Gowda
%            \item Guinean
%            \item Hispanic American
%            \item Honduran
%            \item Huastec
%            \item Hui
%            \item Hungarian
%            \item Hutu
%            \item Indian, Mixed
%            \item Indonesian
%            \item Inupiat
%            \item Iranian
%            \item Israeli
%            \item Italians
%            \item Jamaican
%            \item Jats
%            \item Jews, Ashkenazi
%            \item Jordanian
%            \item Kamma
%            \item Kapu
%            \item Karnataka
%            \item Kayastha
%            \item Khatris
%            \item Koreans
%            \item Kshatriya
%            \item Kurmis
%            \item Lambadi
%            \item Lhoba
%            \item Lingayat
%            \item Lithuanian
%            \item Mahishya
%            \item Maonan
%            \item Mapuche
%            \item Marathas
%            \item Mexican
%            \item Moroccans
%            \item Mulao
%            \item Munda
%            \item Naga
%            \item Namasudra
%            \item Navajo
%            \item Omani
%            \item Oraon
%            \item Otomi
%            \item Pakistani
%            \item Poles
%            \item Polynesians
%            \item Quechua
%            \item Rajput
%            \item Reddy
%            \item Roma
%            \item Sakunapakshollu
%            \item Salvadoran
%            \item Santal
%            \item Saudi
%            \item Scot
%            \item Serb
%            \item Sicilian
%            \item Sikh
%            \item Slovenes
%            \item Sri Lankan
%            \item Swiss
%            \item Tamil
%            \item Tatar
%            \item Tehuelche
%            \item Teli
%            \item Tibetan
%            \item Timorese
%            \item Trinidadian
%            \item Tunisian
%            \item Turks
%            \item Uyghur
%            \item Venezuelan
%            \item Vietnamese
%            \item Wichi
%            \item Yemeni
%            \item Yerukula
%            \item Yugoslav
%            \item Yupik
%            \item Fleming
%        \end{enumerate}
%    \end{multicols}
%}
This gave us over samples from over 147 distinct populations.
For this research, we are only using one of these 147 populations: the Columbians.
The Columbian populace was used as training data to determine the $c, d$ parameters.
There exists plans to utilize more of the data we have collected from ALFRED for verificiation and training.
