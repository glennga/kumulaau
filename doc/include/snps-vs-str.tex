\chapter{Single Nucleotide Polymorphisms vs. Microsatellites}\label{ch:SNPvsMicrosatellites}
In this chapter we briefly cover what polymorphisms, single nucleotide polymorphisms, and microsatellites are.

\section{Polymorphism}\label{sec:polymorphism}
\emph{Polymorphism} is the existence of two or more \emph{morphs} (forms) in some population.
These variations are generated in humans by the mutational process, recombination, and mating.

In terms of genetics, polymorphisms describe different alleles within a gene.
These changes may occur in both coding regions (i.e.\ a polymorphism results in different protein being produced) and
non-coding regions.
Once this change occurs in some individual, all the offspring of said individual will possess this new morph until
another mutation occurs.
Given this nature these changes may also select toward certain alleles, or solely exist as a result of genetic drift.
In most cases though, polymorphisms are not functional and are selectively
neutral~\cite{wrightGeneticVariationPolymorphisms2005}.
Consequently this paper does not review the functional aspect of certain morphs and assumes this selective neutrality.

\section{Single Nucleotide Polymorphisms}\label{sec:singleNucleotidePolymorphism}
\emph{Single nucleotide polymorphisms} (abbreviated as SNPs), are single nucleotide differences in a DNA sequence.
The set of nucleotides that compose such a sequence are $\{ \texttt{A}, \texttt{T}, \texttt{C}, \texttt{G} \}$, which
represent adenine, thymine, cytosine and guanine respectively.
To be considered a SNP and not a point mutation, the least frequent allele must be present in at least 1\% of a
population.
An example of an SNP with 2 variants is given by the set $S$ below.
The differences in this SNP are the \texttt{G} and \texttt{A} in bold.
\begin{align*}
    S &= \{ S_1, S_2 \} \\
    S_1 &= \ldots \texttt{GGGTCACAT\textbf{G}GTTACAGTA} \ldots \\
    S_2 &= \ldots \texttt{GGGTCACAT\textbf{A}GTTACAGTA} \ldots
\end{align*}

Theoretically a SNP could have up to 4 variants, but most are biallelic in nature.
Given that two states exist for most SNPs, multiple SNPs are typically used to determine one's evolutionary
history~\cite{jehanSingleNucleotidePolymorphism2006}.

\section{Microsatellites}\label{sec:microsatellites}
\emph{Microsatellites} are sequences in DNA that are repeated in tandem.
These sequences typically range from 1 to 5 nucleotides long, with copies of this sequence (measured in
\emph{repeat units}) ranging from $\sim 3$ nucleotides to
100~\cite{roseThresholdSizeMicrosatellite1998,fanBriefReviewShort2007}.
An example of a microsatellite variation with the AT sequence and 5 repeat units in arbitrary
sequence of DNA is given below.
The microsatellite variation itself is represented in bold.
\begin{equation*}
     \ldots \texttt{AACG\textbf{ATATATATAT}GGCTA} \ldots
\end{equation*}

In more precise terms, a microsatellite is a set of nucleotide sequences whose elements are only varied by repeat units.
The microsatellite itself belongs to a class of repeat polymorphisms known as short tandem repeat polymorphisms (STRP).
An example of a repeat polymorphism with the repeat sequence CA is given by the set $M$ below.
\begin{align*}
    M &= \{ M_1, M_2, M_3, M_4, M_5, M_6, M_7 \} \\
    M_1 &= \ldots \texttt{GGGT\textbf{CACACACACACACACACA}GTTAC} \ldots \\
    M_2 &= \ldots \texttt{GGGT\textbf{CACACACACACACACA}GTTACAG} \ldots \\
    M_3 &= \ldots \texttt{GGGT\textbf{CACACACACACACA}GTTACAGTA} \ldots \\
    M_4 &= \ldots \texttt{GGGT\textbf{CACACACACACA}GTTACAGTAAA} \ldots \\
    M_5 &= \ldots \texttt{GGGT\textbf{CACACACACA}GTTACAGTAAATA} \ldots \\
    M_6 &= \ldots \texttt{GGGT\textbf{CACACACA}GTTACAGTAAATAGG} \ldots \\
    M_7 &= \ldots \texttt{GGGT\textbf{CACACA}GTTACAGTAAATAGGAA} \ldots
\end{align*}

Microsatellites are of evolutionary interest due to their repeat unit instability across generations.
In the 1990s, microsatellites and other repeat polymorphisms were replaced by nucleotide polymorphisms due to being
more abundant, easier to recover, and having more functional consequences if the sequence is in a coding
region~\cite{graySingleNucleotidePolymorphisms2000,butlerSTRsVsSNPs2007}.
When compared to SNPs though, microsatellites mutate at rates between $10^{-3}$ to $10^{-6}$ mutations per generation,
1 to 10 orders of magnitude greater than SNPs~\cite{gemayelJunkVariableTandemRepeats2012}.
Approximately 40--60 SNPs are required to discriminate as effectively as 13--15 short repeat
polymorphisms~\cite{butlerSTRsVsSNPs2007}.
For these reasons, we have chosen to use microsatellites over SNPs as the genetic marker to explore
these various evolutionary models.
