\section{Single Nucleotide Polymorphisms vs. Microsatellites}\label{sec:snpvm}
In this section we cover what single nucleotide polymorphisms and microsatellites are.

\subsection{Single Nucleotide Polymorphisms}\label{subsec:snp}
\emph{Single nucleotide polymorphisms} (abbreviated as SNPs), are single nucleotide differences in a DNA sequence.
The set of nucleotides that compose such a sequence are $\{ A, T, C, G \}$, which represent adenine, thymine,
cytosine and guanine respectively.
An example of an SNP with 3 variants is given by the set $S$ below:
\begin{align*}
    S &= \{ S_1, S_2, S_3 \} \\
    S_1 &= \ldots \mathit{GGGTCACAT}\textbf{G}\mathit{GTTAC} \ldots \\
    S_2 &= \ldots \mathit{GGGTCACAT}\textbf{C}\mathit{GTTACAG} \ldots \\
    S_3 &= \ldots \mathit{GGGTCACAT}\textbf{A}\mathit{GTTACAGTA} \ldots
\end{align*}

\subsection{Microsatellites}\label{subsec:m}
\emph{Microsatellites} are sequences in DNA that are repeated in tandem.
These sequences typically range from 1 to 5 nucleotides long, with copies of this sequence (in units of
\emph{repeat units}) ranging from $\sim 6$ nucleotides to
100~\cite{roseThresholdSizeMicrosatellite1998,fanBriefReviewShort2007}.
An example of a microsatellite variation with the $AT$ sequence and 5 repeat units in arbitrary
sequence of DNA is given below.
\begin{equation*}
     \ldots \mathit{AACG}\textbf{ATATATATAT}\mathit{GGCTA} \ldots
\end{equation*}

In more precise terms, a microsatellite is a set of nucleotide sequences whose elements are only varied by repeat units.
The microsatellite itself belongs to a class of repeat \emph{polymorphisms} known as repeat polymorphisms.
An example of a repeat polymorphism with the repeat sequence $CA$ is given by the set $M$ below.
\begin{align*}
    M &= \{ M_1, M_2, M_3, M_4 \} \\
    M_1 &= \ldots \mathit{GGGT}\textbf{CACACACACACA}\mathit{GTTAC} \ldots \\
    M_2 &= \ldots \mathit{GGGT}\textbf{CACACACACA}\mathit{GTTACAG} \ldots \\
    M_3 &= \ldots \mathit{GGGT}\textbf{CACACACA}\mathit{GTTACAGTA} \ldots \\
    M_4 &= \ldots \mathit{GGGT}\textbf{CACACA}\mathit{GTTACAGTAAA} \ldots
\end{align*}

Microsatellites are of evolutionary interest due to their repeat unit instability across generations.
In the 1990s, microsatellites and other repeat polymorphisms were replaced by nucleotide polymorphisms due to being
more abundant and having functional consequences if the sequence is in a coding
region~\cite{graySingleNucleotidePolymorphisms2000}.
When compared to SNPs though, microsatellites mutate at rates between $10^{-3}$ to $10^{-6}$ per generation,
1 to 10 orders of magnitudes greater than SNPs~\cite{gemayelJunkVariableTandemRepeats2012}.
Approximately 40--60 SNPs are required to discriminate as effectively as 13--15 short repeat
polymorphisms~\cite{butlerSTRsVsSNPs2007}.
